\chapter{\textit{Business Understanding} (João)}
\label{chap:bus}

\section{Definir Objetivos de Negócio}
Na fase inicial do CRISP-DM, o foco esteve em alinhar os objetivos do projeto com as necessidades práticas, o principal objetivo do projeto é prever a probabilidade de ocorrência de doenças cardíacas com base em dados comportamentais e demográficos. O sucesso do projeto será avaliado com base na precisão dos modelos e das suas previsões.

\section{Avaliar a Situação}
Avaliámos os recursos disponíveis, os requisitos, e os potenciais riscos que poderiamos enfrentar:
\begin{itemize}
    \item \textbf{Dados:} Dataset público BRFSS, fornecido pelo \textit{Centers for Disease Control and Prevention (CDC)} \citep{brfss2021}.
    \item \textbf{Requisitos:} Tratar dados, lidar com dados em falta, normalizar, discretizar, reduzir e preparar o dataset para modelos supervisionados e não supervisionados.
    \item \textbf{Riscos:} Possível baixa qualidade, overfitting e eventuais limitações de generalização dos modelos.
\end{itemize}

\section{Plano e estrutura do Projeto}

Desde o inicio que foi planeada e acordada uma estrutura funcional para o projeto onde os modelos pudessem devolver sempre a mesma estrutura de dados que pudesse depois ser alimentada às funções de visualização, criando assim uma distinção entre o a interface e a lógica.
\begin{itemize}
    \item \textbf{Ferramentas:} Uso de \texttt{Python} em junção com Jupyter Notebooks, bibliotecas como \texttt{scikit-learn} e \texttt{pandas}, e \LaTeX para documentação.
    \item \textbf{Etapas por ordem cronológica:}
    \begin{enumerate}
        \item Análise inicial dos dados e exploração.
        \item Preparação dos dados, incluindo tratamento de valores em falta e normalização.
        \item Desenvolvimento dos modelos de previsão.
        \item Comparação e interpretação dos resultados obtidos.
    \end{enumerate}
\end{itemize}
